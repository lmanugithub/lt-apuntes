\documentclass[10pt,letterpaper]{book}
\usepackage[utf8]{inputenc}
\usepackage[spanish]{babel}
\usepackage{graphicx}
\usepackage{lmodern}
\usepackage{kpfonts}
\usepackage[left=2cm,right=2cm,top=2cm,bottom=2cm]{geometry}
\author{Luis Manuel Guerrero}

\usepackage[table,xcdraw]{xcolor}
\usepackage{longtable}
\begin{document}



\begin{description}
    \item[ADR (American Depositary Receipts):] \hfill \break Recibo de depósito estadounidense, el cual asegura que se encuentran en depósito las acciones de empresas de un país no estadounidense y que están bajo control de una institución bancaria de los Estados Unidos.
    \item[FIBRAS (Fideicomisos de Infraestructura y Bienes Raíces):] \hfill \break Son fideicomisos que se dedican a la adquisición o construcción de bienes inmuebles que se destinan al arrendamiento o a la adquisición del derecho de percibir ingresos provenientes del arrendamiento de dichos inmuebles, así como otorgar financiamiento para esos fines y no dejar de ser vehículos de financiamiento para la adquisición o construcción de bienes raíces. Al inversionista se le eliminan varios riesgos como problemas legales y morosidad con inquilinos, morosidad, etc. Se emiten mediante Certificados Bursátiles Fiduciarios Inmobiliarios (CBFI) y se negocian en la BMV en oferta pública o privada.
    \item[Títulos Referenciados a Acciones (TRACS):] \hfill \break Son la certificación de participación ordinaria que representan el patrimonio de fideicomisos de inversión y que mantienen canastas de acciones o de instrumentos de deuda, estos títulos permiten al inversionista comprar un índice o portafolio de acciones con títulos de deuda a través de un solo título. Se negocian en el mercado como activo subyacente de renta fija y variable.
    \item[Exchange Trade Funds (ETFS):] \hfill \break Son instrumentos internacionales que se negocian en el sistema internacional de cotizaciones (SIC) de la BMV, estos instrumentos pueden estar compuestos como activo subyacente: acciones, títulos de deuda, commodities, instrumentos de derivados. Son instrumentos muy parecidos a los TRACS pero internacional. Se negocian en el mercado como activo subyacente de renta fija, variable y de materias primas.
    \item[WARRANTS:]  \hfill \break Instrumento de derivados que le da al tenedor el derecho a comprar o vender un activo subyacente, a cambio del pago de una prima a cierto precio de ejercicio durante el periodo en donde se obliga al vendedor o comprador a realizar la operación.
    \item[Certificados de Capital de Desarrollo (CKDES):] \hfill \break Títulos destinados para el financiamiento de uno o más proyectos de varios sectores, mediante la adquisición de una o varias empresas en industrias como: minera, bienes inmuebles, desarrollo de tecnología, infraestructura y otras. Son de rendimiento variable a un plazo determinado, no pagan intereses y no cuentan con calificación.
    \item[Notas estructuradas con componente de capital:] \hfill \break Son productos financieros emitidos por instituciones financieras donde se puede o no garantizar el capital invertido y se puede ofrecer un rendimiento determinado con la opción de un rendimiento adicional en el caso que se cumplan las condiciones de mercado, su diseño es complejo ya que es un instrumento en donde hay una combinación de instrumentos de renta fija, con otro de variable por medio de títulos de derivados.
    \item[Financiamiento o deuda (renta fija)] \begin{itemize}
        \item \begin{description}
            \item[Pagaré Financiero] \hfill \break Títulos emitidos por empresas de factoraje o arrendadoras financieras.
        \end{description}        
        \item \begin{description}            
            \item[Pagaré mediano plazo] \hfill \break Títulos emitidos por empresas comerciales o industriales con un interés pagadero en 28 días, por plazo máximo de tres años, estos también operan en el mercado de dinero.
        \end{description}    
    \end{itemize}
    \item[Obligaciones] \hfill \break Títulos de crédito del cual representan la participación de su tenedor en un crédito colectivo a largo plazo (3 – 7 años) a cargo de una sociedad pudiendo ser: \begin{enumerate}
        \item Hipotecarias: aquellas que están garantizadas por la hipoteca sobre bienes inmuebles de la empresa emisora.
        \item Quirografarias: aquellas que se refieren a empresas emisoras reconocidas con solvencia económica y moral.
        \item Convertibles: aquellas en donde se estipula que podrán convertirse en acciones.
    \end{enumerate}
    \item[Certificado de Participación ordinaria (CPO):] \hfill \break Es un instrumento que confiere el derecho de dividendos y liquidación de acciones, sobre acciones de una empresa mexicana que no pueden adquirir directamente los extranjeros y se realiza mediante NAFIN. Para dar cabida a la inversión extranjera hacia el capital de las empresas; el plazo puede ser entre 2 a 3 años.
    \item[Certificados de participación inmobiliaria (CPI):] \hfill \break Son títulos nominativos a largo plazo que dan derecho a una parte proporcional de los bienes inmuebles dados en garantía; son muy útiles para empresas constructoras, arrendadoras y desarrolladoras de inmuebles, del cual constituyen un fideicomiso en una institución fiduciaria, siendo esta la que emite el CPI, colocándose a través de casas de bolsa.
    De acuerdo a la información presentada se puede concluir que estos instrumentos fueron diseñados a raíz de una necesidad del mercado demandante de recursos, cabe destacar que los instrumentos son y serán susceptibles a cambios y reformas que se presenten. Además de surgir otros nuevos de acuerdo a la madurez del mercado.
\end{description}

\textbf{Índices accionarios}
Un índice accionario es un valor de referencia que refleja el comportamiento de un conjunto de acciones, es un instrumento estadístico que ayuda a medir el cambio relativo que experimenta una variable durante un periodo determinado en un mercado, sector o país. Este tipo de índices se convierten en herramientas sencillas de entender y fundamentales para la toma de decisiones por parte de los inversionistas, pues dan una visión global de los movimientos bursátiles. En el mercado de capitales en México se encuentran los siguientes índices:

% Please add the following required packages to your document preamble:
% \usepackage[table,xcdraw]{xcolor}
% Beamer presentation requires \usepackage{colortbl} instead of \usepackage[table,xcdraw]{xcolor}
% \usepackage{longtable}
% Note: It may be necessary to compile the document several times to get a multi-page table to line up properly
\begin{longtable}{|l|l|l|l|}
    \hline
    \rowcolor[HTML]{004783} 
    {\color[HTML]{FFFFFF} \textbf{Índice}} & {\color[HTML]{FFFFFF} \textbf{Descripción}} & {\color[HTML]{FFFFFF} \textbf{\begin{tabular}[c]{@{}l@{}}Revisión y \\ balanceo de la \\ muestra\end{tabular}}} & {\color[HTML]{FFFFFF} \textbf{\begin{tabular}[c]{@{}l@{}}Tamaño de la \\ muestra\end{tabular}}} \\ \hline
    \endfirsthead
    %
    \multicolumn{4}{c}%
    {{\bfseries Table \thetable\ continued from previous page}} \\
    \endhead
    %
    \rowcolor[HTML]{FFFFFF} 
    {\color[HTML]{555555} \begin{tabular}[c]{@{}l@{}}IPC (Índice de \\ precios y \\ cotizaciones)\end{tabular}} & {\color[HTML]{555555} \begin{tabular}[c]{@{}l@{}}Expresa el rendimiento accionario del mercado \\ mexicano de valores en función de la variación \\ de los precios de una muestra balanceada, \\ ponderada y representativa del conjunto de \\ emisoras del BMV.\end{tabular}} & {\color[HTML]{555555} \begin{tabular}[c]{@{}l@{}}Revisión una vez \\ al año y su \\ balance es \\ trimestral.\end{tabular}} & {\color[HTML]{555555} \begin{tabular}[c]{@{}l@{}}35 emisoras a la \\ serie accionaria \\ más bursátil.\end{tabular}} \\ \hline
    \rowcolor[HTML]{FFFFFF} 
    {\color[HTML]{555555} \begin{tabular}[c]{@{}l@{}}IRT (Índice de \\ rendimiento \\ total)\end{tabular}} & {\color[HTML]{555555} Variación del rendimiento total del IPC.} & {\color[HTML]{555555} \begin{tabular}[c]{@{}l@{}}Revisión una vez \\ al año y su \\ balance es \\ trimestral.\end{tabular}} & {\color[HTML]{555555} \begin{tabular}[c]{@{}l@{}}35 emisoras a la \\ serie accionaria \\ más bursátil.\end{tabular}} \\ \hline
    \rowcolor[HTML]{FFFFFF} 
    {\color[HTML]{555555} \begin{tabular}[c]{@{}l@{}}INMEX (Índice \\ México)\end{tabular}} & {\color[HTML]{555555} \begin{tabular}[c]{@{}l@{}}Índice de precios ponderados por valor de \\ capitalización ajustado por acciones flotantes, \\ teniendo la finalidad de establecerse con un \\ valor subyacente para emisión de productos \\ derivados sobre el índice.\end{tabular}} & {\color[HTML]{555555} \begin{tabular}[c]{@{}l@{}}Revisión una vez \\ al año y su \\ balance es \\ trimestral.\end{tabular}} & {\color[HTML]{555555} \begin{tabular}[c]{@{}l@{}}20 emisoras a la \\ serie accionaria \\ más bursátil.\end{tabular}} \\ \hline
    \rowcolor[HTML]{FFFFFF} 
    {\color[HTML]{555555} \begin{tabular}[c]{@{}l@{}}IMC30 (Índice de \\ mediano valor de \\ mercado)\end{tabular}} & {\color[HTML]{555555} \begin{tabular}[c]{@{}l@{}}Su objetivo es constituirse como un indicador \\ altamente representativo y confiable del \\ desempeño de las empresas de valor de \\ mercado mediano y mercado accionario.\end{tabular}} & {\color[HTML]{555555} \begin{tabular}[c]{@{}l@{}}Revisión una vez \\ al año y su \\ balance es \\ trimestral.\end{tabular}} & {\color[HTML]{555555} \begin{tabular}[c]{@{}l@{}}30 emisoras a la \\ serie accionaria \\ más bursátil.\end{tabular}} \\ \hline
    \rowcolor[HTML]{FFFFFF} 
    {\color[HTML]{555555} \begin{tabular}[c]{@{}l@{}}IH (Índice de \\ vivienda Habita)\end{tabular}} & {\color[HTML]{555555} \begin{tabular}[c]{@{}l@{}}Índice dedicado al sector de la vivienda, \\ indicador representativo del mercado accionario, \\ sirviendo como subyacente de productos \\ financieros.\end{tabular}} & {\color[HTML]{555555} \begin{tabular}[c]{@{}l@{}}Revisión una vez \\ al año y su \\ balance es \\ trimestral.\end{tabular}} & {\color[HTML]{555555} No hay límite.} \\ \hline
\end{longtable}




\end{document}